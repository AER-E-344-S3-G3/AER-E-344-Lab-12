\chapter{Conclusion} \label{cp:conclusion}

To analyze the effects of airfoil icing on aerodynamic forces, we used the icing wind tunnel at Iowa State University to collect data on both a clean airfoil and an iced airfoil. From the data, we were able to find the coefficients of lift, drag, and moment and the lift to drag ratio at various \acrshort{aoa}. Analyzing the various coefficients, we concluded that icing on the airfoil generally decreases lift, increases drag, and decreases the moment. For pilots flying in icy condition, this has two potential side effects: unwanted disturbances and higher required thrust to maintain steady flight. Finding ways to mitigate the effects of icing on aircraft not only saves money but may also save lives.